\documentclass[12pt,a4paper]{article}
\usepackage{geometry}

% Page margin layout
\geometry{left=2.3cm,right=2cm,top=2.5cm,bottom=2.0cm}

\usepackage{graphics}
\usepackage{graphicx}
\usepackage{subfigure}
\usepackage{epsfig}
\usepackage{float}
\usepackage{mfirstuc}
\usepackage{hyperref}

\usepackage{booktabs}
\usepackage{threeparttable}
\usepackage{longtable}
\usepackage[ruled,linesnumbered]{algorithm2e}
\usepackage{listings}

% cite package, to clean up citations in the main text. Do not remove.
\usepackage{cite}

\usepackage{color,xcolor}

%% The amssymb package provides various useful mathematical symbols
\usepackage{amssymb}
%% The amsthm package provides extended theorem environments
\usepackage{amsthm}
\usepackage{amsfonts}
\usepackage{enumerate}
\usepackage{enumitem}
\usepackage{listings}

\usepackage{indentfirst}
\setlength{\parindent}{2em} % Make two letter space in the first paragraph
\usepackage{setspace}
\linespread{1.5} % Line spacing setting

\setlength{\parskip}{0.5em} % Paragraph spacing setting

%%%%%%%%%%%%%
\newcommand{\TeamNumber}{2208487}  % Fill your team number here
\newcommand{\Problem}{C}  % Replace your team chosen problem here
\newcommand{\PaperTitle}{To Harvest More: Best Trading Strategies on Gold and Bitcoins}  % Change your paper title here
%%%%%%%%%%%%%

\newcommand{\Predictor}{ARIMA }
\newcommand{\Decision}{Markowitz mean-variance model }
\newcommand{\DecisionAssessment}{Sharp ratio}

%% Page header and footer setting
\usepackage{fancyhdr}
\usepackage{lastpage}
\pagestyle{fancy}
\fancyhf{}
\fancyhead[L]{\texttt{Team \#  \TeamNumber }}
\fancyhead[R]{\texttt{Page {\thepage} of \pageref{LastPage}}}



\begin{document}

%%%%%%%%%%%%%%%%%%%%%%%%%%%%%%%%%%%%%%%%%%%%
\makeatletter % change default title style
\renewcommand*\maketitle{%
    \begin{center} 
        \bfseries  % title 
        {\LARGE \@title \par}  % LARGE typesetting
        \vskip 1em  %  margin 1em
        {\global\let\author\@empty}  % no author information
        {\global\let\date\@empty}  % no date
        \thispagestyle{empty}   %  empty page style
    \end{center}%
  \setcounter{footnote}{0}%
}
\makeatother
%%%%%%%%%%%%%%%%%%%%%%%%%%%%%%%%%%%%%%%%%%%%


%%%%%%%%%%%%%%%%%
% Summary sheet
\begin{table}[h]
\renewcommand{\arraystretch}{1.2}
\label{Summary}
\begin{center}
\resizebox{\textwidth}{!}
{\large
\begin{tabular}{c c c c c c c}
{\,} & \textbf{Problem Chosen} & {\qquad \quad} & \textbf{2022} & {\qquad \quad} & \textbf{Team Control Number} & {\,}\\
{} & \raisebox{-2ex}[0cm][0cm]{\LARGE{\textbf{\textcolor{red}{\Problem}}}} & {} & \textbf{MCM/ICM} & {} & \raisebox{-2ex}[0cm][0cm]{\LARGE{\textbf{\textcolor{red}{\TeamNumber}}}} & {}\\
{} & {} & {} & \textbf{Summary Sheet} & {} & {} & {}\\
\bottomrule
\end{tabular}}
\end{center}
\end{table}
\vskip -1.5em
%%


\begin{spacing}{1.2}   % summary line spacing setting


% Enter your summary here replacing the (red) text
% Replace the text from here ...

Traders buy and sell volatile assets to maximize their income. Recently, with the growing interest of the public in cryptocurrencies, gold and bitcoin become a feasible combo. Our team is requested to determine the trading strategies for a trader which uses only the past daily prices. 

As for problem 1, First we plot the price data and found that they show non-stationarity in the sense of mean. We therefore utilize \Predictor to predict the prices of gold and bitcoin. The predictor shows good capability to accurately predict the price of future.

To find the best timing to sell and buy the two assets, we first rate them with the percentage change, based on predicted value. Then we accomplish position management with Markowitz model. The choice we made yields good income.

As for problem 2, we

As for problem 3, we

Finally


\textbf{Key words:} 

\end{spacing}

\thispagestyle{empty}


%%%%%%%%%%%%%%%%%
\newpage

\title{
\Large{\textcolor{black}{\PaperTitle}}
}




\maketitle



%%%%%%%%%%%%%%%%%
% Table of Contents
\tableofcontents
\setcounter{tocdepth}{2}

%%%%%%%%%%%%%%%%%
\newpage
\setcounter{page}{1}

%% main text

\begin{spacing}{1.2} 



%%%%%%%%%%%%%%%%%
\section{Introduction}
\label{Problem_Statement}

\subsection{Problem Background}
With the popularity of cryptocurrencies and the simplification of trading methods, more of the population become market traders. Some of them expect to outperform inflation, while others want to create wealth. By buying and selling volatile assets frequently, market traders pursue a goal to maximize their total return. Gold and bitcoins enjoy great popularity these days for their complementary characteristics in risk and value. Gold is stable in price and has lower risk while the value of bitcoins varies greatly and thus has a higher risk, as is shown in Figure \ref{figure:prices2in1}.

\begin{figure}[H]
	\centering{\includegraphics[width=17.0cm]{figures/prices_2in1.png}}
	\caption{Gold and bitcoin daily prices, in U.S. dollars per troy ounce and U.S. dollars per bitcoin. \textbf{Source:} London Bullion Market 
		Association, 9/11/2021 and NASDAQ, 9/11/2021 }
	\label{figure:prices2in1}
\end{figure}

 Regarding trading rules, Gold is only traded on days the market is open while bitcoins are traded every day. Commissions are charged to make each transaction. For market traders to achieve their goals, they need to build a model to determine the strategies to manage their portfolios well.

\subsection{Restatement of the Problem}

\begin{itemize}
	\item Develop a model that gives the best daily trading strategy based only on price data up 
	to that day, and calculate how much the initial \$1000 investment is worth on 9/10/2021 using the 
	model and strategy.
	
	\item Present evidence that your model provides the best strategy.
	
	\item Determine how sensitive the strategy is to transaction costs and analyze how the costs
	affect the strategy and results.
	
\end{itemize}

\subsection{Our Approach}

\begin{enumerate}
	\item \textbf{To predict the prices of bitcoin and gold and make decision, we use \textit{\Predictor} (Autoregressive integrated moving average).}
	To predict future prices with existing data is a difficult problem because too many factors may influence the prices. The international situation, national policies, and even social media can have a considerable impact on the prices. Moreover, the data shows non-stationarity. To take as many factors as possible and predict accurately according to their inner laws, we adopt this approach to predict the prices, which is proved to give satisfying results.
	
	\item \textbf{To make decision on trading, we adopt \textit{\Decision} and \textit{\DecisionAssessment}. }
\end{enumerate}

%%%%%%%%%%%%%%%%%
\section{Assumptions and Justifications}
\label{Assumptions_Justifications}

\subsection{Assumptions}
To simplify the problem stated above, we make following assumptions, each of which is justified properly: 
\begin{enumerate}
	
	\item \textbf{The trader does not have a bias towards a lower risk.} The two given assests, gold and bitcoin, differs a lot in risk. To simplify the problem, we assume that the trader will not prefer lower risk than higher and only cares about higher income.
	
	\item \textbf{The trader will have \$1000 in the beginning, and the transaction commissions for gold and bitcoin are $\alpha_{gold}=1\%$ and $\alpha_{bitcoin}=2\%$, respectively.}


	\item \textbf{The market trader sells all of the gold and bitcoins by the end of the five-year trading period, i.e. on 9/10/2021.} Generally, investors cares about funding liquidity. Among cash, gold and bitcoins, only cash can circulate unhindered in the market. So we make this assumption and thus measure the outcome in cash.
	
\end{enumerate}


\subsection{Symbols and Definitions}



\subsection{Symbols and Definitions}





%%%%%%%%%%%%%%%%%


\section{Data Preprocessing}
\label{DataPreprocessing}


\subsection{Metric Calculating}

\subsubsection{Average Change}

Using the trained predictor, we calculate the average of change in the period of 5,10,15,20,and 25 days.

The change each day is defined as

$$
Percentage \ change = PC = \frac{New \ price-Old \ price}{Old \ price} \times 100\%
$$


The result for the given data is in the \nameref{sec:AppendixFT}. Figure \ref{figure:bitcoin_ac} and Figure \ref{figure:gold_ac} shows the average change we calculated for bitcoin and gold, respectively.


\subsubsection{Bias}

Bias measures how much the closing price shifts from the average price.

The formula to calculate the Bias in a $n$ day period is

$$
Bias = \frac{Closing \ price}{The \ mean \ price \ of \ n \ days} - 1
$$

Bias can help to track the market for the raise and fall of the price, which help us to decide whether and when to sell or buy.

\subsubsection{Moving Averages (MA)}

An $n$ day's MA is the average of the price today and the previous $n-1$ days. Plotting all the MAs in a chart, it can reveal the trend of the price of the given period. Combined with current price, it helps us to find the favorable timing to increase our outcome

\section{Mathematical Modeling}
\label{MathModels}

\subsection{\Predictor Predictor}

\Predictor is widely used for forecasting time series data, which is a generalization of ARMA (autoregressive moving average) model. ARMA requires the mean function of the data to be stationary, so ARIMA take an initial differencing step to eliminate the non-stationarity of the mean function. In this case, we find the first order difference is stationary, so we take first order difference. 

To apply the model, we first have to test the stationarity.

\subsubsection{Augmented Dickey-Fuller Test}

An augmented Dickey–Fuller test (ADF) tests the null hypothesis that a unit root is present in a time series sample. it can test whether the data is stationary or trend-stationary.

\subsubsection{Finite Difference}

\subsubsection{Choice of Parameters}

\subsubsection{Residual Test and Durbin-Watson Test}
To test whether our model has extracted all information in the data, we use

\subsubsection{Predict the Change Using \Predictor}

\subsection{Trade Decision}

\subsubsection{Rating Gold and Bitcoin}

\subsubsection{Position Management}


%%%%%%%%%%%%%%%%%
\section{Results and Solutions}
\label{Results_Solutions}

Result analysis

Discussions



%%%%%%%%%%%%%%%%%
\section{Sensitivity Analysis}
\label{SensitivityAnalysis}

For each transaction, a commission is charged. The commission can also have an impact on our model. We tested our model with different commissions and found that the level of commission does not significantly affect our model.

%%%%%%%%%%%%%%%%%
\section{Strength and Weakness}
\label{Strength_Weakness}


\subsection{Strength}

The models have the following strengths:

\begin{itemize}
\item \textbf{Our model yields considerable income}

\item \textbf{Our model do not involve artificial neural network, which saves computational resources.}

\item \textbf{Our model build up a good system to assess a financial asset.}

\item \textbf{Our model can be generalized to all problems in trading.}

\end{itemize}


\subsection{Weakness}

Though our model performs well, the models have following weaknesses:

\begin{itemize}
\item \textbf{We do not consider factors other than prices.}

\item \textbf{The choice of our model's parameters relies on experience and grid search.}
\end{itemize}


%%%%%%%%%%%%%%%%%
\section{Conclusions}
\label{Conclusions}

In this paper, we build a \Predictor based predictor to enhance the income of trading cash, gold, and bitcoins. First, we preprocess the data and fill the missing values. Then, we finish the feature engineering by finite difference and feature store to boost the machine learning effect. After that, we use a RNN-based \Predictor to predict the price. Finally, we reach the optimality of our model by compare different trading strategies. Sensitive tests and evaluation are carried out. We measured our model by annualized return, retracement and sharp rate. The influence of commissions are experimented to test the sensitivity. Our model yields satisfying results.




%%%%%%%%%%%%%%%%%
\newpage
\begin{thebibliography}{00}

%% \bibitem{label}
%% Text of bibliographic item

\bibitem{Pierre2020}
Pierre Rostan,Alexandra Rostan,Mohammad Nurunnabi. Options trading strategy based on ARIMA forecasting[J]. PSU Research Review,2020.


\bibitem{Marcos2018}
Marcos Lopez dai Prado, \textit{Advances in Financial Machine} Learning[M]:75-82. ISBN 978-1119482086

\bibitem{Li2017}
Li F. Modelling the stock market using a multi-scale approach[D]. University of Leicester, 2017.

\bibitem{Pang2005}
Marling H, Emanuelsson S. The Markowitz Portfolio Theory[J]. November, 2012, 25: 2012.


\end{thebibliography}
\addcontentsline{toc}{section}{References}


\addtocounter{page}{-1}
\thispagestyle{empty}

%%%%%%%%%%%%%%%%%
\newpage
\addtocounter{page}{-1}
\thispagestyle{empty}

{\centering\section*{Memorandum to the Trader}}

Considering the intensely changing financial markets and the difficulty of handling your portfolio, using an appropriate model to make predictions and strategies to trade are of vital importance to improve income. It is a great honor for us to develop the model for you to buy, hold and sell your assets. Here is our model based on \textbf{\Predictor} and strategies for you to trade your assets effectively.  

\begin{enumerate}
	\item 
	
	\item
	
	\item
\end{enumerate}

We appreciate this opportunity to help you to build up a trading strategy for cash, gold and bitcoins, and we firmly believe that our model can be utilized in maximizing your total income. Feel free to contact us for further information on the proposal.

Sincerely yours

\textit{MCM 2022 Team}



\end{spacing}


%%%%%%%%%%%%%%%%%
\newpage
%% The Appendices part is started with the command \appendix;
%% appendix sections are then done as normal sections
\appendix
\addtocounter{page}{-1}
\thispagestyle{empty}

\section*{Appendix: Figures and Tables}
\label{sec:AppendixFT}

\begin{figure}[H]
	\centering{\includegraphics[width=17.0cm]{figures/bitcoin_ac.png}}
	\caption{ Average Change in 5,10,15,20,25 days for bitcoin}
	\label{figure:bitcoin_ac}
\end{figure}

\begin{figure}[H]
	\centering{\includegraphics[width=17.0cm]{figures/gold_ac.png}}
	\caption{ Average Change in 5,10,15,20,25 days for gold }
	\label{figure:gold_ac}
\end{figure}


\end{document}


%% End of template
